\documentclass[10pt,a4paper]{article}

\usepackage{./MGLassignment}
\newcommand{\sol}[1]{#1} % #1 for solution
\newcommand{\optional}[1]{#1} % #1 for showing optional

% Anomalies
%\setlength{\textheight}{26.5cm}

\title{\textbf{$1^o$ Σετ Ασκήσεων Θεωρίας Υπολογισμού \\  }}
\author{\textbf{Αντώνιος Ραφαήλ Ελληνιτάκης, 2017030118} \\
σε συνεργασία με: \\
Καποιος Καπου Καποτε}
\date{20 Μαρτίου 2021}


\begin{document}
	
\maketitle

\newpage
	
\opening
%{University}
{Πολυτεχνείο Κρήτης}
%{Dept}
{Σχολή Ηλεκτρονικών Μηχανικών και Μηχανικών Υπολογιστών}
%{Course}
{ΠΛΗ 402 -- Θεωρία Υπολογισμού -- 2021} % - \textlatin{Theory of Computation}}
%{Term}
{Διδάσκων: Μ. Γ. Λαγουδάκης}
%{Assignment}
{1η Σειρά Ασκήσεων}
%{Deadline}
{Επίλυση μέχρι: \textbf{04/04/2021}} \vspace{-5ex}

\sol{
\begin{center}
\huge\bf Λύσεις
\end{center}
\vspace*{-0,5cm}
}

\boldmark{1. Κανονικές Εκφράσεις}
\noindent

%%%%%%%%%%%%%%%%%%%%%%%%%%%%%%%%%%%%%%%%%%%%%
%         Εκφώνηση 1                        %
%%%%%%%%%%%%%%%%%%%%%%%%%%%%%%%%%%%%%%%%%%%%%
\textbf{1.1 [15\%]}
Γράψτε Κανονικές Εκφράσεις για τις παρακάτω γλώσσες. \\
\textbf{(α)} $L = \big\{ w \in \{a,b\}^*:\text{ η $w$ περιέχει ακριβώς 1 εμφάνιση του $b$ και άρτιο αριθμό απο $a$}\big\}$. \\
\sol{
	\hspace*{0.5cm} {\em \textbf{Απάντηση:} }
	Η ζητούμενη κανονική έκφραση είναι η παρακάτω:
	\\
	$$ ((aa)^*b(aa)^*) \cup (aba(aa)^*)  $$ \\
	Το πρώτο καλύπτει όλα τα ενδεχόμενα, εκτός από αυτό του να είναι το $b$ ο δεύτερος χαρακτήρας, το οποίο καλύπτεται απ΄0 την δεύτερη περίπτωση. \\
	%your answer here
}
\textbf{(β)} $L = \big\{ w \in \{a,b\}^*:\text{ η $w$ αρχίζει και τελειώνει με διαφορετικό σύμβολο, και το πρώτο έχει άρτιο πλήθος.}\big\}$. \\
\sol{
	\hspace*{0.5cm} {\em \textbf{Απάντηση:} }
	\\
	Χωρίζουμε το πρόβλημα σε δύο περιπτώσεις. Αυτή που η συμβολοσειρά ξεκινάει απο $a$ και αυτή που ξεκινάει απο $b$.
	\begin{itemize}
		\item Για συμβολοσειρά που ξεκινάει απο το $a$(και άρα τελειώνει στο $b$): 
		$$ a(ab^*a \cup b)^*b  $$
		\item Για συμβολοσειρά που ξεκινάει απο το $b$(και άρα τελειώνει στο $a$): 
		$$ b(ba^*b \cup a)^*a  $$
		\item Για να συγχωνέψουμε τους δύο περιορισμούς, χρησιμοποιούμε την ένωση:
		$$ a(ab^*a \cup b)^*b \cup b(ba^*b \cup a)^*a  $$
	\end{itemize}
	%your answer here
}
\textbf{(γ)} $L = \big\{ w \in \{a,b\}^*:\text{ το πλήθος των $b$ στην $w$ είναι } 3k+2 (k \geq 2)\big\}$. \\
\sol{
	\hspace*{0.5cm} {\em \textbf{Απάντηση:} }
	\\ 
	Για την υλοποίηση της κανονικής έκφρασης, χρησιμοποιήθηκε ένα αυτόματο, το οποίο γίνεται καλύτερα εικόνα στο μάτι. Αρχικά ορίζουμε ένα αυτόματο με 3 καταστάσεις, μία για κάθε υπόλοιπο με την διαίρεση με το 3. Ώς τελική κατάσταση ορίζουμε αυτήν που αντιπροσωπεύει υπόλοιπο 2, καθώς ο αριθμός ου θέλουμε είναι $3k+2$. 
	\begin{center}	
		Αρχικά, έχουμε το αυτόματο: \\
		\begin{tikzpicture}[->,>=stealth',shorten >=0pt,node distance=2cm,auto]
			\en
			\node[state,initial] (a)				{q$0$};  % this is on the top left, and it is a starting state
			\node[state] (b) [below of = a] {q$1$};  % define where the other nodes will be presented
			\node[state,accepting] (c) [right of = b] {q$2$};  % accepting means final state.
						
			\foreach \from/\to in {a/b, a/c, b/c}
			% structure of an arrow:
			%\draw[->] (from_node) [path_of_the_arrow] [edge] [node {text}] (to_node)
			\draw[->] (a) [bend right=10] edge node {$b$}  (b);  % "node {...}" just adds the specifier on the arrow.
			\draw[->] (a) [loop above] edge node {$a$} (a);
			\draw[->] (b) [bend right=10] edge node {$b$} (c);
			\draw[->] (b) [loop below] edge node {$a$} (b);
			\draw[->] (c) [bend right=10] edge node {$b$} (a);
			\draw[->] (c) [loop below] edge node {$a$} (c);
		\end{tikzpicture}
		
		'Επειτα απο την πρώτη απλοποίηση, γίνεται: \\
		\begin{tikzpicture}[->,>=stealth',shorten >=0pt,node distance=2cm,auto]
			\en
			\node[state,initial] (a){q$0$};
			\node[state] (b) [below of = a] {q$1$};
			\node[state,accepting] (c) [right of = b] {q$2$};
			
			\foreach \from/\to in {a/b, a/c, b/c}
			\draw[->] (a) [bend right=10] edge node {$b$}  (b);
			\draw[->] (a) [loop above] edge node {$a$} (a);
			\draw[->] (b) [bend right=10] edge node {$b$} (c);
			\draw[->] (b) [loop below] edge node {$a$} (b);
			\draw[->] (c) [loop above] edge node {$a^*ba^*b$} (c);   
			\draw[->] (c) [loop below] edge node {$a$} (c);
		\end{tikzpicture}
		
		Συνεχίζουμε τις απλοποιήσεις(τo $q0$ μπορεί να πάει στον εαυτό του είτε μέσω του $(a^*ba^*b)$ είτε μέσω του $a$ ): \\
		\begin{tikzpicture}[->,>=stealth',shorten >=0pt,node distance=2cm,auto]
			\en
			\node[state,initial] (a){q$0$};
			\node[state] (b) [below of = a] {q$1$};
			\node[state,accepting] (c) [right of = b] {q$2$};
			
			\foreach \from/\to in {a/b, a/c, b/c}
			\draw[->] (a) [bend right=10] edge node {$b$}  (b);
			\draw[->] (a) [loop above] edge node {$a$} (a);
			\draw[->] (b) [bend right=10] edge node {$b$} (c);
			\draw[->] (b) [loop below] edge node {$a$} (b);
			\draw[->] (c) [loop above] edge node {$(a^*ba^*b) \cup a$} (c);   
		\end{tikzpicture}
		
		Λίγο πριν το τέλος, μπορούμε να απαλείψουμε την κατάσταση $q1$ γράφοντας ώς κανονική έκφραση την διαδρομή απο το $q0$ έως το $q2$¨\\
		\begin{tikzpicture}[->,>=stealth',shorten >=0pt,node distance=2cm,auto]
			\en
			\node[state,initial] (a){q$0$};
			\node[state,accepting] (c) [right of = b] {q$2$};
			
			\foreach \from/\to in {a/b, a/c, b/c}
			\draw[->] (a) [bend right=10] edge node {$a^*ba^*b$}  (c);
			\draw[->] (c) [loop below] edge node {$(a^*ba^*b) \cup a$} (c);   
		\end{tikzpicture}
	\end{center}
	Με βάση το παραπάνω αυτόματο και την (απλή) απλοποίηση του, βγαίνει η κανονική έκφραση:
	$$a^*ba^*b((a^*ba^*b) \cup a)^*$$
}

\end{document}

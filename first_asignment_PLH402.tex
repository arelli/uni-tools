\documentclass[10pt,a4paper]{article}

\usepackage{subfig}
\usepackage[demo]{graphicx}


\usepackage{./MGLassignment}
\newcommand{\sol}[1]{#1} % #1 for solution
\newcommand{\optional}[1]{#1} % #1 for showing optional

% Anomalies
%\setlength{\textheight}{26.5cm}

\title{\textbf{$1^o$ Σετ Ασκήσεων Θεωρίας Υπολογισμού \\  }}
\author{\textbf{Αντώνιος Ραφαήλ Ελληνιτάκης, 2017030118} \\
σε συνεργασία με: \\
Καποιος Καπου Καποτε}
\date{20 Μαρτίου 2021}


\begin{document}
	
\maketitle

\newpage
	
\opening
%{University}
{Πολυτεχνείο Κρήτης}
%{Dept}
{Σχολή Ηλεκτρονικών Μηχανικών και Μηχανικών Υπολογιστών}
%{Course}
{ΠΛΗ 402 -- Θεωρία Υπολογισμού -- 2021} % - \textlatin{Theory of Computation}}
%{Term}
{Διδάσκων: Μ. Γ. Λαγουδάκης}
%{Assignment}
{1η Σειρά Ασκήσεων}
%{Deadline}
{Επίλυση μέχρι: \textbf{04/04/2021}} \vspace{-5ex}

\sol{
\begin{center}
\huge\bf Λύσεις
\end{center}
\vspace*{-0,5cm}
}

\boldmark{Κανονικές Εκφράσεις}
\noindent

%%%%%%%%%%%%%%%%%%%%%%%%%%%%%%%%%%%%%%%%%%%%%
%         Εκφώνηση 1                        %
%%%%%%%%%%%%%%%%%%%%%%%%%%%%%%%%%%%%%%%%%%%%%
\textbf{1 [15\%]}
Γράψτε Κανονικές Εκφράσεις για τις παρακάτω γλώσσες. \\
\textbf{(α)} $L = \big\{ w \in \{a,b\}^*:\text{ η $w$ περιέχει ακριβώς 1 εμφάνιση του $b$ και άρτιο αριθμό απο $a$}\big\}$. \\
\sol{
	\hspace*{0.5cm} {\em \textbf{Απάντηση:} }
	Η ζητούμενη κανονική έκφραση είναι η παρακάτω:
	\\
	$$ ((aa)^*b(aa)^*) \cup (aba(aa)^*)  $$ \\
	Το πρώτο καλύπτει όλα τα ενδεχόμενα, εκτός από αυτό του να είναι το $b$ ο δεύτερος χαρακτήρας, το οποίο καλύπτεται απ΄0 την δεύτερη περίπτωση. \\
	%your answer here
}
\textbf{(β)} $L = \big\{ w \in \{a,b\}^*:\text{ η $w$ αρχίζει και τελειώνει με διαφορετικό σύμβολο, και το πρώτο έχει άρτιο πλήθος.}\big\}$. \\
\sol{
	\hspace*{0.5cm} {\em \textbf{Απάντηση:} }
	\\
	Χωρίζουμε το πρόβλημα σε δύο περιπτώσεις. Αυτή που η συμβολοσειρά ξεκινάει απο $a$ και αυτή που ξεκινάει απο $b$.
	\begin{itemize}
		\item Για συμβολοσειρά που ξεκινάει απο το $a$(και άρα τελειώνει στο $b$): 
		$$ a(ab^*a \cup b)^*b  $$
		\item Για συμβολοσειρά που ξεκινάει απο το $b$(και άρα τελειώνει στο $a$): 
		$$ b(ba^*b \cup a)^*a  $$
		\item Για να συγχωνέψουμε τους δύο περιορισμούς, χρησιμοποιούμε την ένωση:
		$$ a(ab^*a \cup b)^*b \cup b(ba^*b \cup a)^*a  $$
	\end{itemize}
	%your answer here
}
\textbf{(γ)} $L = \big\{ w \in \{a,b\}^*:\text{ το πλήθος των $b$ στην $w$ είναι } 3k+2 (k \geq 2)\big\}$. \\
\sol{
	\hspace*{0.5cm} {\em \textbf{Απάντηση:} }
	\\ 
	Χωρίζουμε το πρόβλημα σε δύο κομμάτια: ένα είναι να έχω $3k$ $b$ μή συνεχόμενα, και το άλλο είναι να έχω επιπλέον δύο $b$ επίσης μη συνεχόμενα.
	\begin{itemize}
		\item Για το πρώτο μέρος, έχω:
		$$a^*ba^+ba^+ba^*$$
		\item Για το δεύτερο:
		$$ba^+ba^*$$
	\end{itemize}
	
	Τελικά καταλήγω με την παρακάτω υλοποίηση, με \en Kleene Star \gr σε όλη την πρώτη πρόταση, καθώς μπορεί να χρησιμοποιηθεί όσες φορές θέλουμε, και με θετική ολοκλήρωση στο τελευταίο $a$ καθώς είναι απαραίτητο για να μας διαχωρίζει απο το προτελευταίο $b$.
	$$(a^*ba^+ba^+ba^+)^*ba^+ba^*$$
}


\boldmark{Κανονικές Εκφράσεις}
\noindent
\textbf{2 [20\%]}
Κατασκευάστε πεπερασμένα αυτόματα (με μονοψήφιο αριθμό καταστάσεων) για τις παρακάτω γλώσσες:
\\
\textbf{(α)} $L = \big\{ w \in {a,b,c}^*: \text{ οι εμφανήσεις του $b$ είναι άρτιες και του $c$ περιττές}\}$\\
\sol{
	\hspace*{0.5cm} {\em \textbf{Απάντηση:} }
	Για να ξεκινήσω την λύση, απαριθμώ τις καταστάσεις που με ενδιαφέρουν: \\
	\begin{center}
	\begin{tabular}{|c|c|c|}
		\hline
		\textbf{$b$ περιττό} & \textbf{$c$ περιττό} & $a$ \en Dont Care \gr \\
		\hline
		\textbf{$b$ περιττό} & \textbf{$c$ άρτιο} & $a$ \en Dont Care \gr \\
		\hline
		\textbf{$b$ άρτιο} &  \textbf{$c$ περιττό} & $a$ \en Dont Care \gr  \\
		\hline
		\textbf{$b$ άρτιο} & \textbf{$c$ άρτιο} & $a$ \en Dont Care \gr  \\
		\hline
		... & ... &  ...  \\
		\hline
	\end{tabular}
	\end{center}
	
	Χωρίζω τα παραπάνω σε 4 διακριτές καταστάσεις, \textbf{\en ObEc=$q_1$(odd b even c), EbEc=$q_0$,EbOc=$q_3$ \gr και \en ObOc=$q_2$\gr, με τελική κατάσταση το \en EbOc \gr.}
	\\
	\begin{center}
	\begin{tikzpicture}[->,>=stealth',shorten >=0pt,node distance=2cm,auto]
		\en
		\node[state,initial] (EbEc)				{q$0$};  % this is on the top left, and it is a starting state
		\node[state] (ObEc) [below of = EbEc] {q$1$};  % define where the other nodes will be presented
		\node[state,] (ObOc) [right of = ObEc] {q$2$};  % accepting means final state.
		\node[state,accepting] (EbOc) [above of = ObOc] {q$3$};
		
		%\foreach \from/\to in {a/b, a/c, b/c}
		% structure of an arrow:
		%\draw[->] (from_node) [path_of_the_arrow] [edge] [node {text}] (to_node)
		% Draw the arrows from the EbEc node:
		\draw[->] (EbEc) [bend right=30] edge node {$c$}  (EbOc); 
		\draw[->] (EbEc) [loop above] edge node {$a$}  (EbEc);
		\draw[->] (EbEc) [bend right=30] edge node {$b$}  (ObEc);
		% Draw the arrows from the ObEc node:
		\draw[->] (ObEc) [loop below] edge node {$a$}  (ObEc);
		\draw[->] (ObEc) [bend right=30] edge node {$b$}  (EbEc);
		\draw[->] (ObEc) [bend right=30] edge node {$c$}  (ObOc);
		% Draw the arrows from the EbOc node:
		\draw[->] (EbOc) [loop above] edge node {$a$}  (EbOc);
		\draw[->] (EbOc) [bend right=30] edge node {$b$}  (ObOc);
		\draw[->] (EbOc) [bend right=30] edge node {$c$}  (EbEc);
		% Draw the arrows from the ObOc node:
		\draw[->] (ObOc) [loop below] edge node {$a$}  (ObOc);
		\draw[->] (ObOc) [bend right=30] edge node {$b$}  (EbOc);
		\draw[->] (ObOc) [bend right=30] edge node {$c$}  (ObEc);

	\end{tikzpicture}
	\end{center}
	%ΛΥΣΗ
}


\end{document}

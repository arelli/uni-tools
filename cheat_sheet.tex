\documentclass[12pt, a4paper]{report}  % report depends on what Im writing
\usepackage[utf8]{inputenc}
\usepackage{graphicx}
\usepackage{amsmath}  % supports many of the math that I wil need
\graphicspath{ {/} }

\title{Control Systems First Lab}
\author{Rafail Ellinitakis, Apostolos Gioumertakis, Vagelis Katsoupis}
\date{March 2021}

\begin{document}

\maketitle  % using maketitle we print the title in the standard format

%%%%%%%%%%%%%%%%%%%%%%%%%%
%       ABSTRACT         %
%%%%%%%%%%%%%%%%%%%%%%%%%%

\begin{abstract}
This is a simple paragraph at the beginning of the 
document. A brief introduction about the main subject.
\end{abstract}


This is a test.
													% The double empty line prompts
													% LaTeX to create a paragraph afterwards.
Some of the \textbf{greatest}
discoveries in \underline{science} 
were made by \textbf{\textit{accident}}.

%%%%%%%%%%%%%%%%%%%%%%%%%%
%      GRAPHICS          %
%%%%%%%%%%%%%%%%%%%%%%%%%%

%\includegraphics{plot2}  % an image, taken as is
\begin{figure}[h]  % an image or graphic, with a caption and a label and custom size
    \centering
    \includegraphics[width=1\textwidth]{plot2}
    \caption{a nice plot}
    \label{fig:mesh1}
\end{figure}

%As you can see in the figure \ref{fig:mesh1}, the 
%function grows near 0. Also, in the page \pageref{fig:mesh1} 
%is the same example.

%%%%%%%%%%%%%%%%%%%%%%%%%%%
%         LISTS           %
%%%%%%%%%%%%%%%%%%%%%%%%%%%

\begin{itemize}
  \item The individual entries are indicated with a black dot, a so-called bullet.
  \item The text in the entries may be of any length.
\end{itemize}

\begin{enumerate}
  \item This is the first entry in our list
  \item The list numbers increase with each entry we add
\end{enumerate}

In physics, the mass-energy equivalence is stated 
by the equation $E=mc^2$, discovered in 1905 by Albert Einstein.  % the $....$ operator tells us that its an inline mathematical expression

The mass-energy equivalence is described by the famous equation
\[ E=mc^2 \]
discovered in 1905 by Albert Einstein.   % use the \[ ... \] operator to state a display mathem. expresison

%%%%%%%%%%%%%%%%%%%%%%%%%%%%%%%%%%%
%             M A T H             %
%%%%%%%%%%%%%%%%%%%%%%%%%%%%%%%%%%%
Subscripts in math mode are written as $a_b$ and superscripts are written as $a^b$. These can be combined an nested to write expressions such as

\[ T^{i_1 i_2 \dots i_p}_{j_1 j_2 \dots j_q} = T(x^{i_1},\dots,x^{i_p},e_{j_1},\dots,e_{j_q}) \]
 
We write integrals using $\int$ and fractions using $\frac{a}{b}$. Limits are placed on integrals using superscripts and subscripts:

\[ \int_0^1 \frac{dx}{e^x} =  \frac{e-1}{e} \]

Lower case Greek letters are written as $\omega$ $\delta$ etc. while upper case Greek letters are written as $\Omega$ $\Delta$.

Mathematical operators are prefixed with a backslash as $\sin(\beta)$, $\cos(\alpha)$, $\log(x)$ etc.

%%%%%%%%%%%%%%%%%%%%%%%%%%%%%%%%%
%            CHAPTERS           %
%%%%%%%%%%%%%%%%%%%%%%%%%%%%%%%%%

\chapter{First Chapter}

\section{Introduction}  % marks the beginning of a new section.
						% Section numbering can be disabled by including a *, like \section*{}.
This is the first section.


\section{Second Section}

This is the second one

\subsection{First Subsection}
First subsection of the second station

\section*{Unnumbered Section}
This is an unnumbered section


%%%%%%%%%%%%%%%%%%%%%%%%%%%%%%%
%        TABLES               %
%%%%%%%%%%%%%%%%%%%%%%%%%%%%%%%
% table without borders(you can use https://www.tablesgenerator.com/ to create LaTeX complex tables)
\begin{center}
\begin{tabular}{ c c c }  % tabular is the default LATEX method to create tables. {c c c}tells LATEX that there will be three columns and that the text inside each one of them must be centred. You can also use r to align the text to the right and l for left alignment. 
 cell1 & cell2 & cell3 \\ 
 cell4 & cell5 & cell6 \\  
 cell7 & cell8 & cell9    
\end{tabular}
\end{center}  % wrap in center, so that it appears centered.

%table with borders
\begin{center}
\begin{tabular}{ |c|c|c| }  % |c|c|c| declares that there are three columns, separated by a vertical line
 \hline  % this inserts a horizontal line
 cell1 & cell2 & cell3 \\ 
 cell4 & cell5 & cell6 \\ 
 cell7 & cell8 & cell9 \\ 
 \hline  % bottom horizontal line
\end{tabular}
\end{center}

\end{document}
